
\documentclass{article}
\usepackage{amsmath} 
\usepackage[affil-it]{authblk}

\title{Cubical Complexes} % Sets article title
\author{Enei Sluga, Simon Hehnen} % Sets authors name
\affil
{
    Faculty of Computer and Information Science \\
    University of Ljubljana % Your institution for the title page
}
\date{\today} % Sets date for date compiled

\begin{document}
    \maketitle

    \section{Introduction}
    In the realm of computation, many datasets are inherently discrete and structured in a
    grid-like arrangement. Examples of such datasets include images and videos,
    where we typically work with pixels organized on a two-dimensional grid.
    When analyzing the structure of this data and constructing a simplicial complex,
    traditional triangulations may not always be the most effective approach.
    Instead, leveraging the pixels as fundamental building blocks could offer a
    more suitable alternative.
    
    An additional advantage is that, unlike the floating-point values used in triangulations,
    we work with discrete coordinates on a grid. This discretization can provide a
    computational benefit, as we consistently construct cubical complexes in a fixed,
    ordered manner, potentially offering increased efficiency over triangulation-based methods.
    
    \section{Generating Cubical Complexes}
    From the script we know that a \textbf{cubical complex} $K$ on an $n \times n$ grid
    is a collection of cubes such that $\sigma \in K$ and $\tau \subseteq \sigma$, then $\tau \in K$.
    In this case $n \in \mathbf{N}$ refers to the number of square along each side.
    Our grid has size $n \times n$. This means the collection of all simplices has the following attributes:
    \begin{enumerate}
        \item 0-dimensional cubes: $(n+1)^2$ verticies
        \item 1-dimensional cubes: $2n(n+1)$ edges
        \item 2-dimensional cubes: $n^2$ squares
    \end{enumerate}



    \section{Computing Homology}
    TODO explain how homolgy is computed

    \section{Persistant Homology}
    TODO persistant Homology

    \section{Conclusion}
    TODO Conclusion

    \section{Division of Work}
    Simon did the code for the creation of cubical complexes and constructed the examples.
    Enei provided the code for Homology computation.
    Persistant Homology was done by both together with the examples.

    \section{Bibliography}*
    We used the section on cubical complexes from the lecture script (P.106-108).

\end{document}
